%%%%%%%%%%%%%%%%%%%%%%%%%%%%%%%%%%%%%%%%%%%%%%%%%%%%%%%%%%%%%%%%%%%%%%%%%%%%%%%%
%% 서울대학교 데이터마이닝연구실 구성원들의 박사학위논문 작성을 위해 아래 저작자의 자료를 일부 수정하였습니다.
%% Author: zeta709 (zeta709@gmail.com) 
%%%%%%%%%%%%%%%%%%%%%%%%%%%%%%%%%%%%%%%%%%%%%%%%%%%%%%%%%%%%%%%%%%%%%%%%%%%%%%%%
%% 2018-01-08 산업공학과 데이터마이닝 전공을 산업공학과로 수정하고, 영어 석사 논문 작성 시 편리한 작업을 위한 추가 수정 및 팁 작성 by 문지형 jhmoon@dm.snu.ac.kr

\RequirePackage{fix-cm} 
% oneside : 단면 인쇄용
% twoside : 양면 인쇄용
% ms: 석
% phd : 박사
% openright : 챕터가 홀수쪽에서 시작
\documentclass[twoside,ms]{snuthesis_utf8}

%%%%%%%%%%%%%%%%%%%%%%%%%%%%%%%%%%%%%%%%
%% 목차 양식을 변경하는 코드
%% subfigure (subfig) package 사용 여부에 따라
%% tocloft의 옵션을 다르게 지정해야 한다.
%\usepackage[titles,subfigure]{tocloft} % when you use subfigure package
\usepackage[titles]{tocloft} % when you don't use subfigure package
\makeatletter % don't delete me
\renewcommand\cftchappresnum{Chapter~}
\renewcommand\cftfigpresnum{Figure~}
\renewcommand\cfttabpresnum{Table~}


\usepackage[pdftex,bookmarks=true]{hyperref}
\usepackage{tabularx}
\usepackage{array,multirow,graphicx,rotating,booktabs}
\usepackage{caption}  %subfigure
\usepackage{subcaption}  %subfigure
\usepackage[round, sort, numbers]{natbib} %reference style
\newcommand\mycite[1]{[\citenum{#1}]}
\newcommand\myauthor[1]{\citeauthor{#1} (\citeyear{#1})}
\usepackage{pbox} % line break in table
\usepackage{adjustbox}
\usepackage{footnote}
\usepackage{color}
\usepackage{colortbl}
\usepackage{amsmath}
\usepackage{float} % position here
\usepackage{lscape} % for landscape page
\usepackage{kotex}

\makeatother % don't delete me
\newlength{\mytmplen}
\settowidth{\mytmplen}{\bfseries\cftchappresnum\cftchapaftersnum}
\addtolength{\cftchapnumwidth}{\mytmplen}
\settowidth{\mytmplen}{\bfseries\cftfigpresnum\cftfigaftersnum}
\addtolength{\cftfignumwidth}{\mytmplen}
\settowidth{\mytmplen}{\bfseries\cfttabpresnum\cfttabaftersnum}
\addtolength{\cfttabnumwidth}{\mytmplen}
%% 목차 양식을 변경하는 코드 끝
%%%%%%%%%%%%%%%%%%%%%%%%%%%%%%%%%%%%%%%%

%%%%%%%%%%%%%%%%%%%%%%%%%%%%%%%%%%%%%%%%
%% 다른 패키지 로드
%% http://faq.ktug.or.kr/faq/pdflatex%B0%FAlatex%B5%BF%BD%C3%BB%E7%BF%EB
%% 필요에 따라 직접 수정 필요
\ifpdf
	% \input glyphtounicode\pdfgentounicode=1 %type 1 font사용시
	%\usepackage[pdftex,unicode]{hyperref} % delete me
	%\usepackage[pdftex]{graphicx}
	%\usepackage[pdftex,svgnames]{xcolor}
\else
	%\usepackage[dvipdfmx,unicode]{hyperref} % delete me%
	%\usepackage[dvipdfmx]{graphicx}
	%\usepackage[dvipdfmx,svgnames]{xcolor}
\fi
%%%%%%%%%%%%%%%%%%%%%%%%%%%%%%%%%%%%%%%%
%
%% \title : 22pt로 나오는 큰 제목
%% \title* : 16pt로 나오는 작은 제목
\title{Dissertation Title Dissertation Title \\abcdefghijk}
\title*{석사학위논문 한글제목}

\titlen{Dissertation Title Dissertation Title abcdefghijk}

\author{Gildong~Hong}
\author*{홍길동} % Same as \author.
\authorn{홍~길~동}
\phonenumber{010-1111-1111}
\studentnumber{2016-11111}
\advisor{Nada~Ga}
\advisor*{조성준}
\advisorn{조 성 준}
\graddate{2018~년~~2~월}
\submissiondate{2018~년~~2~월}
\submissiondaten{2018~년~~2~월~~1~일}
\approvaldate{2017~년~~12~월}

\committeemembers%
{교 수 님}%
{조 성 준}%
{교 수 님}%
{교 수 님}%
{교 수 님} %

%% Length of underline
\setlength{\committeenameunderlinelength}{5cm}

\begin{document}
\pagenumbering{Roman}
\makefrontcover
\makeapproval

%agreement page
%\cleardoublepage
%\makeagreement

\cleardoublepage
\pagenumbering{roman}

\keyword{SNU, dmlab, thesis}
\begin{abstract}
In this dissertation, ...

\end{abstract}
\cleardoublepage

% % 여기 수정할 것
\tableofcontents
\addcontentsline{toc}{chapter}{\contentsname}
\cleardoublepage

\listoftables
\addcontentsline{toc}{chapter}{\listtablename}
\cleardoublepage

\listoffigures
\addcontentsline{toc}{chapter}{\listfigurename}
\cleardoublepage

\pagenumbering{arabic}

\chapter{Introduction}

This guide is made for graduates who are unfamiliar with graduate thesis latex templates.
I added some tips to reduce working time and use latex more conveniently.

\clearpage
\chapter{Tips}
\section{Image Insertion}
It is convenient to create a folder named 'figure' for image insertion.
Once the tex file is compiled, a lot of dirty files will be created, therefore, it is quite messy without specific image folder.


\begin{figure}[h]
	\centering
	\includegraphics[width=1\textwidth]{figure/div.png}  % scale보다는 문서크기를 보고 width의 비율로 정하는 편이 크기 조정할 때 직관적이다. 예시 그림의 경우에는 원래 가로 여백이 있었다.
	\caption{Data, Insight, Value}
	\label{fig:div}
\end{figure}


\clearpage
\section{Table insertion}
\subsection{Basic}
\url{https://www.tablesgenerator.com/latex_tables} creates the most basic template!

\clearpage
\subsection{Advanced}
The most annoying part of latex work is table insertion.
Size modification, highlights, and annotation in table are very annoying compared to Hancom, so many people turn to it.
However, latex will feel much easier if you learn only the following introductions for graduation thesis.

\textit{adjustbox}: It adjusts the overall size of the table. If not, a table may be generated  beyond the document scope.

\textit{columncolor}: It shades the entire column to emphasize the results of my model.

\textit{footnotemark and footnotetext}: Inside the table, the $\setminus$footnote does not work. 
For this reason, footnotemark and footnotetext are used.
Note that the page on which the annotation exists is not always the same (...) as the table.
When the footnotemark is used several times inside the table, the annotation number becomes strange.
In this case, use \textit{addtocounter}.
\newline
\newline
If you need more than this, let's do googling.

\clearpage
\begin{table}[h]
\centering
	\caption{Average error rate on bAbI story-based tasks with 10k training samples}
	\label{table:babi_result}
\begin{adjustbox}{max width=0.95\textwidth}
\begin{tabular}{l|llllllll>{\columncolor[gray]{0.8}}l}
\hline
Task                                 & \multicolumn{1}{c}{MemNN} & \multicolumn{1}{c}{MemN2N} & \multicolumn{1}{c}{GMemN2N} & \multicolumn{1}{c}{DMN} & \multicolumn{1}{c}{DMN+} & \multicolumn{1}{c}{DNC} & \multicolumn{1}{c}{EntNet\footnotemark}& \multicolumn{1}{c}{RN\footnotemark} & \multicolumn{1}{c}{RMN} \\ \hline
1: Single Supporting Fact            & \textbf{0.0}                       & \textbf{0.0}                        & \textbf{0.0} & \textbf{0.0}                     & \textbf{0.0}                      & \textbf{0.0}                     & \textbf{0.1}                        & \textbf{0.0}                    & \textbf{0.0}                     \\
2: Two Supporting Facts              & \textbf{0.0}                       & \textbf{0.3}                        & \textbf{0.0}                         & \textbf{1.8}                     & \textbf{0.3}                      & \textbf{0.4}                     & \textbf{2.8} & 8.3                    & \textbf{0.5}                     \\
3: Three Supporting Facts            & \textbf{0.0}                       & 9.3                        & \textbf{4.5                        } & \textbf{4.8}                     & \textbf{1.1}                      & \textbf{1.8}                     & 10.6                       & 17.1                  & 14.7                     \\
4: Two Argument Relations            & \textbf{0.0}                       & \textbf{0.0}                        & \textbf{0.0} & \textbf{0.0}                     & \textbf{0.0}                    & \textbf{0.0}                    & \textbf{0.0} & \textbf{0.0}                    & \textbf{0.0}                   \\
5: Three Argument Relations          & \textbf{2.0 }                      & \textbf{0.6}                        & \textbf{0.2 } & \textbf{0.7}                     & \textbf{0.5}                     &\textbf{0.8}                     &\textbf{0.4}                       & \textbf{0.7}                   & \textbf{0.4}                     \\
6: Yes/No Questions                  & \textbf{0.0}                       & \textbf{0.0}                        & \textbf{0.0} & \textbf{0.0}                     & \textbf{0.0}                      & \textbf{0.0}                     & \textbf{0.3}                       & \textbf{0.0}                    & \textbf{0.0}                     \\
7: Counting                          & 15.0                      & \textbf{3.7}                       & \textbf{1.8}                         & \textbf{3.1}                     & \textbf{2.4}                      & \textbf{0.6}                    & \textbf{0.8}                        &\textbf{0.4}                &\textbf{0.5}                    \\
8: Lists/Sets                        & 9.0                       & \textbf{0.8}                        & \textbf{0.3}                         & \textbf{3.5}                     & \textbf{0.0}                     & \textbf{0.3}                     &\textbf{0.1} & \textbf{0.3}                   & \textbf{0.3}                     \\
9: Simple Negation                   & \textbf{0.0}                       & \textbf{0.8}                        & \textbf{0.0}                         & \textbf{0.0 } & \textbf{0.0}                      & \textbf{0.2}                     &\textbf{ 0.0 }                       & \textbf{0.0}                   &\textbf{0.0}                    \\
10: Indefinite Knowledge             & \textbf{2.0}                       & \textbf{2.4}                      & \textbf{0.2} &\textbf{2.5}                     & \textbf{0.0}                      & \textbf{0.2}                     & \textbf{0.0}                        & \textbf{0.0}                    & \textbf{0.0}                   \\
11: Basic Coreference                & \textbf{0.0}                       & \textbf{0.0}                        & \textbf{0.0}                         & \textbf{0.1}                    & \textbf{0.0}                      & \textbf{0.0}                     & \textbf{0.0}                        & \textbf{0.4}                  & \textbf{0.5}                    \\
12: Conjunction                      & \textbf{0.0}                       & \textbf{0.0}                        & \textbf{0.0}                         & \textbf{0.0}                     &\textbf{0.0}                      &\textbf{0.0}                     & \textbf{0.0}                        & \textbf{0.0}                   & \textbf{0.0}                     \\
13: Compound Coreference             & \textbf{0.0}                       & \textbf{0.0}                        & \textbf{0.0} & \textbf{0.2}                     & \textbf{0.0}                      & \textbf{0.1} & \textbf{0.0}                        & \textbf{0.0}                 & \textbf{0.0}                     \\
14: Time Reasoning                   & \textbf{1.0}                       &\textbf{0.0}                        & \textbf{0.0}                         &\textbf {0.0}                     & \textbf{0.0}                      & \textbf{0.4} & \textbf{3.6}                      &\textbf{0.0}                    & \textbf{0.0}                    \\
15: Basic Deduction                  & \textbf{0.0}                       & \textbf{0.0}                        &\textbf{0.0}                         & \textbf{0.0}                    &\textbf{0.0}                    & \textbf{0.0}                     &\textbf{0.0}                        & \textbf{0.0}                   &\textbf{0.0}                     \\
16: Basic Induction                  &\textbf{0.0}                       & \textbf{0.4}                        & \textbf{0.0}                         & \textbf{0.6}                   & 45.3                     & 33.1                    & 52.1                       & \textbf{4.9}                   & \textbf{0.9}                    \\
17: Positional Reasoning             & 35.0                        & 40.7                       & 27.8                        & 40.4                    & \textbf{4.2}                      & 12.0                    & 11.7                       & \textbf{1.6}                    & \textbf{0.3} \\
18: Size Reasoning                   & \textbf{5.0}                      & 6.7                        & 8.5                         & \textbf{4.7}                     & \textbf{2.1}                     & \textbf{0.8}                     & \textbf{2.1}                        & \textbf{2.1}                    & \textbf{2.3}                     \\
19: Path Finding                     & 64.0                      & 66.5                       & 31.0                        & 65.5                    & \textbf{0.0}                      & \textbf{3.9}                     & 63.0                       & \textbf{3.2}                    & \textbf{2.9}                     \\
20: Agent's Motivations              & \textbf{0.0}                       & \textbf{0.0}                        & \textbf{0.0} & \textbf{0.0}                    &\textbf{ 0.0}                      & \textbf{0.0}                     & \textbf{0.0}                        &\textbf{0.0}                    & \textbf{0.0}                     \\ \hline
Mean error (\%)                      & 6.7                       & 6.6                        & 3.7                         & 6.4                     & 2.8                      & 2.7                     & 7.4                        & 2.0                    & \textbf{1.2}                     \\
Failed tasks (err. \textgreater  5\%) & 4                         & 4                          & 3                           & 2                       & \textbf{1}                        & 2                       & 4                          & 2                       & \textbf{1}                       \\ \hline
\end{tabular}
\end{adjustbox}
\end{table}

\addtocounter{footnote}{-1}
\footnotetext{For a fair comparison, we report EntNet's result~\mycite{henaff2016tracking} which was jointly trained on all tasks. It was written in the appendix of the paper.}
\addtocounter{footnote}{+1}
\footnotetext{Our implementation. The result is different from what \myauthor{santoro2017simple} mentioned, which is caused by the initialization~\mycite{santoro2017simple}.}


\chapter{Reference insertion}
\section{Insert reference right before the period mark}
Do not use conventional $\sim\setminus$cite\{\}.

ex: blah blah~\cite{bishop2006}.
\newline
\newline
Because of ( ), I defined `mycite' to appear reference number with [ ].

ex: blah blah~\mycite{bishop2006}.


\clearpage
\section{Insert reference to the author name}
Sometimes you need to put a reference to the author name. 
To do this, I defined `myauthor'.

ex: \myauthor{bishop2006} said blah blah~\mycite{bishop2006}.



\begin{bibpage}
	\bibliographystyle{plainnat}
	\bibliography{ref}
\end{bibpage}


\appendix
\chapter{Appendix title 1}
The appendix does not appear in the table of contents.
In this case, open a .toc file (or click the table of contents page with holding ctrl) and add a line below the bibliography.

\contentsline {chapter}{Bibliography}{8}{section*.9} % 현재 Bibliography
\contentsline {chapter}{Appendix}{9}{section*.10} % {9}는 페이지 번호이므로 알아서 맞추자
\contentsline {section}{\numberline {A}Appendix title 1}{9}{section.Alph0.1}
\contentsline {section}{\numberline {B}Appendix title 2}{10}{section.Alph0.2}

\chapter{Appendix title 2}

\keywordalt{서울대학교, 데이터마이닝연구실, 석사학위논문}
\begin{abstractalt}
한글 요약 내용이 여기에 들어갑니다.
\end{abstractalt}

\acknowledgement
Thanks!

\end{document}

